
\section{Complex numbers}
Här är målet främst att introducera dsl som koncept och även introducera hur vi behandlar komplexa tal eftersom detta kommer återkomma mycket i texten. Vi gör detta genom att snabbt repetera komplexa tal. Vi antar dock att läsaren igentligen är bekväm med komplexa tal och att detta istället är en mjukstart för dsl, haskell och våra egna betäckningar.\\




\begin{code}
data Complex = Complex (double, double)
  deriving Eq
  
whatsReal :: Complex -> double
whatsReal Complex (real,_) = real

whatsImaginary :: Complex -> double
whatsImaginary Complex (_,ima) = ima

  
add :: Complex -> Complex -> Complex
add Complex (r1,i1) Complex (r2,i2) = Complex (r1+r2,i1+i2)

subtract :: Complex -> Complex -> Complex
subtract Complex (r1,i1) Complex (r2,i2) = Complex (r1-r2,i1-i2)

sub :: Complex -> Complex -> Complex
sub complex1 complex2 = subtract complex1 complex2

absolute :: Complex -> double
absolute Complex (real,ima) = root ((real*real)+(ima*ima))

abs :: Complex -> double
abs complex = absolute complex

argument :: Complex -> double
%% TODO


multiply :: Complex -> Complex -> Complex
multiply Complex (r1,i1) Complex (r2,i2)
  = Complex (r1*r2-i1*i2,r1*i2+r2*i1)

\end{code}
