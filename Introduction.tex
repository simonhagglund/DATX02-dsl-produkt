\section{Introduction}
%This is supplementary material intended for students of introductory Control Theory. It is based off of, and aimed to complement, the course ``Control Theory'' (ERE103) given at Chalmers University of Technology, but is meant to be useful as complementary material for any introductory course on the subject.

This is a supplementary learning material for the course ``Control theory'' (ERE103) at Chalmers school of technology. The purpose of this material is to give students at the Computer department a more familiar introduction to the subject of control theory. This is accomplished by creating domain-specific languages (DSLs) for the course. A DSL is a programming language specialised for a domain, in this case control theory. For students who already attended the course ``DSLsofMath'' this approach should be nothing new.  All DSLs in this material are written in Haskell, so the syntax should be familiar from previous courses.


In this material you will find different DSLs for different parts of the subject. Each DSL is developed to explain the different parts of the subject as easy as possible. If you already are familiar with a part of this subject, feel free to move on to the next section.
We hope you find this reading useful. Best of luck!


