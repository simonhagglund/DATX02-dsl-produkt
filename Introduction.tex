\section{Introduction}
%This is supplementary material intended for students of introductory Control Theory. It is based off of, and aimed to complement, the course ``Control Theory'' (ERE103) given at Chalmers University of Technology, but is meant to be useful as complementary material for any introductory course on the subject.

This is a supplementary learning material for the course ``Control theory'' (ERE103) at Chalmers school of technology. The purpose of this material is to give computer science students a more familiar introduction to the subject of control theory. While aimed towards students taking the course ERE103, this learning material is suitable for everyone with an interest in learning more about control theory and is familiar with haskell. If you want to learn haskell just to be able to follow this learning material or just brush up your skills we strongly reccomend learnyouahaskell by Miran Lipovača. 

The method we use in this material is to create domain-specific languages (DSLs) for the course. A DSL is a programming language specialised for a domain, be it math, physics, astronomy, etc. In our case, we will introduce a new DSL for each section, each specialised for the content of said section. For students who already attended the course ``DSLsofMath'' this approach should be nothing new. For students who have not, additional details are provided throughout the first chapter; ''Prerequisites''.

All DSLs in this material are written in Haskell, so the syntax should be familiar from previous courses. If you are not familiar with Haskell or just want to brush up your skills, we recommend ''Learn you a Haskell for Great Good!'' by Miran Lipovača (\href{http://learnyouahaskell.com/}{http://learnyouahaskell.com/}). 

\paragraph{} We hope you find this reading useful. Best of luck!
