
\section{Laplace transform}
The Laplace transform is a tool that enables us to look at a function or equation from a different perspective. More specifically, it takes a function of time and transforms it into a function of frequency.
The Laplace transform is often used to solve differential equations, but more on that later. 

The definition of the Laplace transform is  
\begin{equation*}
    \Lplc \left\{ f(t)\right\}(s) = \int_0^\infty e^{-s\tau} f(\tau) \, \text{d} \tau.
\end{equation*}
When applicable we call the Laplace transform of a function by the capital version of the original functions name, e.g. the Laplace transform of $f(t)$ is $F(s)$.

What are the types? An initial inspection allows us to identify
\begin{code}
t      :: Real 
f, exp :: Real -> Real 
\end{code}
and from DSLsofmath\todo[color=other]{Add citation, probably around page 60} we know that 
\begin{code}
integ :: Real -> Real -> (Real -> Real) -> Real -- definite integral \end{code}% better type for it? 
i.e. the integral takes two real arguments (the limits) and a function, returning a real number. 
Letting $s$ be a real number as well works fine, but in fact it's possible to be a bit more general and let it be a complex number. 

From this we can read that the Laplace transform should have the type
\begin{code}
laplace :: (Real -> Real) -> (Complex -> Complex) 
\end{code}
i.e. it's a function taking a function and transforming it into another function. 

The definition of the Laplace transform is clunky to actually work with, so common practise is to have a table of common functions and their Laplace transforms, and use some rules to calculate the harder problems. We will utilize this tabular approach. 

One of the most important properties of the Laplace transform is that
\begin{equation*}
    \Lplc \left\{ f'(t)\right\} (s) = s F(s) - f(0).
\end{equation*}
This allows us to transform a differential equation into an algebraic equation, i.e. one that only includes the typical operations of algebra.


If we define 
\begin{code}
type TimeDomain      = Real
type FrequencyDomain = Complex
\end{code}
then we can define the type of the laplace transform accordingly: 
\begin{code} 
laplace :: TimeDomain -> FrequencyDomain 
\end{code}








\todo[inline,color=other]{Introduce what a laplace transform is in words.
Show how we construct it in a dsl. (show code)
Some examples (just showing that the code works)
Some examples from the book. As many different ones as possible, if some examples doesn't work with our code explain why.
}
