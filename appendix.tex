\section{Table of Laplace transforms}
% very much work in progress 
\begin{table}[h!]
    \centering
    \caption{Laplace Transforms and rules}
\begin{tabular}{|c | c|}
    \hline 
     f(t) & F(s) \\
    \hline 
     $a f(t) + b g(t)$ & $a F(s) + b G(t)$\\
     \hline 
     $f'(t)$ & $s F(s) - f(0)$ \\
     \hline 
     $e^{\alpha t}$ & $\frac{1}{s-\alpha}$\\
     \hline 
\end{tabular}
    \label{tab:laplacetrans}
\end{table}


\section{Analysis (/Calculus?)} 
\todo[inline,color=other]{Added some preliminary tables for analysis, maybe we wanna write something more?}
\subsection{Types}
\textbf{TODO:} 
\begin{itemize}
    \item Type of derivative, integral 
    \item Higher order functions? (+, $\cdot$, \dots)
\end{itemize}

\subsection{Derivatives}

\begin{table}[h!]
    \centering
    \caption{Derivatives}
    \begin{tabular}{|c|c|}
    \hline 
         $f(x)$ & $f'(x)$ \\
    \hline 
         $x^a$ & $ax^{a-1}$ \\
         $e^x$ & $e^x$ \\
         $e^{kx}$ & $ke^{kx}$ \\
         $a^{x}$ & $a^x \ln a $ \\
         $\sin x$ & $\cos x$ \\
         $\cos x$ & $-\sin x$ \\
         $\tan x$ & $\frac{1}{\sin} x$ \\
         $\arcsin x$ & $\frac{1}{\sqrt{1-x^2}}$ \\
         $\arccos x$ & $-\frac{1}{\sqrt{1-x^2}}$\\
         $\arctan x$ & $\frac{1}{1+x^2}$
         \\ \hline
    \end{tabular}
    \label{tab:derivatives}
\end{table}

\subsection{Partial derivatives}

\subsection{Integrals} 
\begin{table}[h!]
    \centering
    \caption{Integrals}
    \begin{tabular}{|c|c|}
    \hline 
         $f(x)$ & $F(x)$ \\
    \hline 
         $x^a$       & $\frac{x^{a+1}}{a+1}$ \\
         $e^x$       & $e^x$ \\
         $e^{kx}$    & $\frac{e^{kx}}{k}$ \\
         $a^{x}$     & $\frac{a^x}{\ln a}$ \\
         $\sin x$    & $-\cos x$ \\
         $\cos x$    & $\sin x$ \\
         $\tan x$    & $-\ln \abs{\cos x}$ \\
         $\arcsin x$ & $x \arcsin x + \sqrt{1-x^2}$ \\
         $\arccos x$ & $x \arccos x + \sqrt{1-x^2}$ \\
         $\arctan x$ & $\frac{1}{2} \left[2x \arctan x - \ln (1 + x^2)\right]$
         \\ \hline
    \end{tabular}
    \label{tab:integrals}
\end{table}

\section{Code}\label{sec:appcode}
\subsection{}\label{sec:appcodezipwithl}
\begin{code}[caption={Extension of \texttt{zipWith}. If lists are equal in length \texttt{zipWithL} and \texttt{zipWith} are identical. When one list has exhausted their elements, \texttt{i} is used instead}. \texttt{zipWithL k f [a0,a1,a2] [b0,b1,b2,b3,b4]} becomes \texttt{[a0 `f` b0, a1 `f` b1, a2 `f` b2, i `f` b3, i `f` b4].}]
--       null-val zip-function   left   right  zipped
zipWithL :: a -> (a -> a -> b) -> [a] -> [a] -> [b]
zipWithL i f (l:left) (r:right) = f l r : zipWithL i f left right
zipWithL _ _ []       []        = []
zipWithL i f []       (r:right) = f i r : zipWithL i f []   right
zipWithL i f (l:left) []        = f l i : zipWithL i f left []
\end{code}
