%\section{Table of Laplace transforms}\label{sec:applap}
% very much work in progress 

\section{Code}\label{sec:appcode}
%\label{sec:appcodezipwithl}
\begin{codefig}
\begin{minted}{haskell}
--       null-val zip-function   left   right  zipped
zipWithL :: a -> (a -> a -> b) -> [a] -> [a] -> [b]
zipWithL i f (l:left) (r:right) = f l r : zipWithL i f left right
zipWithL _ _ []       []        = []
zipWithL i f []       (r:right) = f i r : zipWithL i f []   right
zipWithL i f (l:left) []        = f l i : zipWithL i f left []
\end{minted}
\caption{Extension of \cmd{zipWith}. If lists are equal in length \cmd{zipWithL} and \cmd{zipWith} are identical. When one list has exhausted their elements, \cmd{i} is used instead. \cmd{zipWithL k f [a0,a1,a2] [b0,b1,b2,b3,b4]} becomes \cmd{[a0 `f` b0, a1 `f` b1, a2 `f` b2, i `f` b3, i `f` b4].}}
\label{sec:appcodezipwithl}
\end{codefig}

%\label{sec:appcodemaxfreq}
\begin{codefig}
\begin{minted}{haskell}
maxfreq :: (Ord a, Num a) => Signal a -> a
maxfreq (Sin _ f _) = f
maxfreq (Scale _ a) = maxfreq a
maxfreq (Deriv   a) = maxfreq a
maxfreq (Integ   a) = maxfreq a
maxfreq (Sum a b) | af < bf   = bf
                  | otherwise = af where
                    af = maxfreq a
                    bf = maxfreq b
\end{minted}
\caption{\cmd{maxfreq} is a helper function on the \cmd{Signal} data type for the function \cmd{simplifySignal} (see code snippet \ref{code:simplifycont}). It returns the maximal frequency found in the given \cmd{Signal} expression tree defined by \cmd{Sum} constructors as tree nodes and \cmd{Sin} as leaves.}
\label{sec:appcodemaxfreq}
\end{codefig}

\begin{codefig}
\caption{\cmd{ComplexNumbers}, a DSL for complex numbers, which is used as a building block for the Laplace transform. Made by \cite{tssarbete}, with slight modifications by us to ensure it fit our DSL in \ref{sec:complex}.}
\label{code:complex}
\begin{code}
module ComplexNumbers where
import GHC.Real

type R = Double

-- Komplexa tal kan ses som ett par av reella värden.

data Complex = Complex (R, R)
    deriving Eq

-- Där första komponenten representerar realdelen och den andra komponenten
-- imaginärdelen.

realPart :: Complex -> R
realPart (Complex (re, im)) = re

imPart :: Complex -> R
imPart (Complex (re, im)) = im

conjugate :: Complex -> Complex
conjugate z = Complex ((realPart z), (negate (imPart z)))

{-
   Argumentet av ett komplext tal.
   Bökigt värre på grund av tråkiga kvadranter och bös,
   men genom att använda funktionen atan2 så blir det mycket snyggare!
-}

arg :: Complex -> R
arg z = atan2 (imPart z) (realPart z)

{-
   Utan atan2 hade vi fått skriva så här:

   arg :: Complex -> R
   arg z  | (imPart z) < 0  && (realPart z) < 0 = atan (imPart z / realPart z) - pi
          | (realPart z) < 0                    = atan (imPart z / realPart z) + pi
          | otherwise                           = atan (imPart z / realPart z)

-}

scale :: R -> Complex -> Complex
scale a z = Complex ((a * realPart z), (a * imPart z))

{-
   j är det komplexa talet med realdelen 0 och imaginärdelen 1
   Många matematiktexter kallar detta talet också för `i`
-}

j :: Complex
j = Complex (0, 1)

-- TODO: Försök få   read . printComplex = id genom att göra printComplex
-- enklare
-- TODO: "name and reuse" something like showIm im = show im ++ "j" to make it
-- easy to change the spacing (or change "j" to "i", or the type from String
-- to ShowS) in only one place

instance Show Complex where
    show = printComplex

printComplex :: Complex -> String
printComplex z
  | r == 0    = imj
  | im == 0   = show r
  | im < 0    = show r ++ "-" ++ show (abs im) ++ "j"
  | otherwise = show r ++ "+" ++ imj
    where im  = imPart z
          r   = realPart z
          imj
            | im == 0  = "0"
            | otherwise = show im ++ "j"

-- | Skapar ett komplext tal utifrån en vinkel.
euler :: R -> Complex
euler phi = Complex ((cos phi), (sin phi))

instance Num Complex where
-- Plus är förhållandevist trivialt
  z + w = Complex ((realPart z + realPart w), (imPart z + imPart w))

-- Multiplikationen följer ifrån att vi multiplicerar komponenterna
-- parvis med varandra (i likhet med (a+b) * (c+d))
  z * w = Complex ((realZ*realW - imZ*imW), (realZ*imW + realW*imZ))
    where realZ = realPart z
          realW = realPart w
          imZ   = imPart z
          imW   = imPart w

-- Negationen av ett komplext tal är ett komplext tal där båda komponenter är negerade

  negate z = Complex ((negate $ realPart z), (negate $ imPart z))

-- Den naturliga generaliseringen av signum från reella till komplexa tal:

  signum z = z / abs z

-- Absolutbeloppet av ett komplext tal är pythagoras sats på dess komponenter.

  abs z = Complex (hyp, 0)
    where hyp  = sqrt (r*r + im*im)
          r    = realPart z
          im   = imPart z

-- Heltal är bara komplexa tal utan imaginärdel

  fromInteger n = Complex (fromInteger n, 0)

instance Fractional Complex where
-- Division utförs genom att man förlänger hela bråket med nämnarens konjugat
  z / w = Complex (realPart zw' / realPart ww', imPart zw' / realPart ww')
    where zw' = z * (conjugate w)
          ww' = w * (conjugate w)

-- En kvot av två heltal är helt reell och därför kommer imaginärdelen vara 0
  fromRational z = Complex (re, 0)
    where re = fromInteger (numerator z) / fromInteger (denominator z)

instance Floating Complex where
-- Det komplexa talet pi är ett tal med realdelen pi och imaginärdelen 0
  pi = Complex (pi, 0)

{-
   Potenslagarna ger att e^(a+bj) <=> e^a * e^bj och
   Eulers formel ger att e^bj <=> cos b + jsin b
-}
  exp z = scale (exp (realPart z)) (euler (imPart z))

{-
   Eulers formel ger att man kan skriva om sin x som (e^jx - e^-jx)/2j
   Eftersom vi definerat exponentialfunktionen för komplexa tal kan vi använda
   eulers formel som vår sinus implementation för komplexa tal
-}
  sin z = (exp (j*z) - exp (-(j*z)))/(scale 2 j)

-- Eulers formel ger att man kan skriva om cos x som (e^jx + e^-jx)/2
  cos z = (exp (j*z) + exp (-(j*z)))/2

  cosh z = Complex (cosh (realPart z) * cos (imPart z),
                    sinh (realPart z) * sin (imPart z))

  sinh z = Complex (sin (realPart z) * cosh (imPart z),
                    cos (realPart z) * sinh (imPart z))

{-
   Eulers formel ger att z = re^(j*phi) och eftersom log och exponentialfunktionen
   är varandras inverser och logaritmlagarna ger ex. `log (a*b)` <=> `log a + log b`.
   Därför kan vi skriva log `z = log r + log (e^(j*phi))` = `log r + j*phi`
-}
  log z = Complex (logR (abs z), arg z)
    where logR = log . realPart

{-
  Funktioner vi troligen / förhoppningsvis inte kommer behöva och som vi därför lämnar
  odefinierade än så länge.
-}
  atanh = undefined
  atan  = undefined
  asinh = undefined
  asin  = undefined
  acosh = undefined
  acos  = undefined
\end{code}
\end{codefig}


\begin{codefig}
\caption{Expression datatype.} \label{code:expression-datatype}
\begin{code}
data Expression a where
  Const      :: a -> Expression a
  Id         :: Expression a
  Impulse    :: Expression a
  Pi         :: Expression a
  (:+:)      :: Expression a -> Expression a -> Expression a 
  (:*:)      :: Expression a -> Expression a -> Expression a 
  (:-:)      :: Expression a -> Expression a -> Expression a 
  (:/:)      :: Expression a -> Expression a -> Expression a
  Conv       :: Expression a -> Expression a -> Expression a
  Shift      :: a -> Expression a 
  Negate     :: Expression a
  Exp        :: Expression a
  Sin        :: Expression a
  Cos        :: Expression a
  Tan        :: Expression a
  Asin       :: Expression a
  Acos       :: Expression a
  Atan       :: Expression a
  Integral   :: Expression a -> a -> Expression a -- function and initial value, i.e. int_y0^t f(x) dx 
  Derivative :: Expression a -> a -> Expression a -- function and initial value, i.e. f' and f(0)
 deriving (Eq, Show)
\end{code}
\end{codefig}

%\iffalse 
% för slutredovisning 
\begin{code}
data Expr x where
  Id         :: Expr a 
  Const      :: x         -> Expr x 
  --            f         -> f(0) -> f' 
  Derivative :: Expr x    -> x    -> Expr x
  ... 
\end{code}

\begin{code}
  Expression         %\textrm{Matematik}% 
  Id                 %$f(x) = x$%
  Const c            %$f(x) = c$%
  Derivative e e0    %\textrm{Derivatan av} e \textrm{med initialvärde} e0%
  
\end{code}


%\fi 


\begin{codefig}
\caption{Function \cmd{eval}. Takes a syntactic representation and turns it into a semantic function.}\label{code:eval-def}
\begin{code}
shift :: Num a => a -> (a -> a) -> (a -> a)
shift tau f = \t -> f (t-tau)

eval :: (Num a, Floating a, Eq a) => Expression a -> a -> a
eval (Const c)           = \t -> c
eval Id                  = id
eval Impulse             = undefined -- TODO
eval Pi                  = eval (Const pi)
eval (e1 :+: e2)         = eval e1 + eval e2
eval (e1 :*: e2)         = eval e1 * eval e2
eval (e1 :-: e2)         = eval e1 - eval e2
eval (e1 :/: e2)         = eval e1 / eval e2
eval (DefInteg e1 e2 e3) = undefined -- TODO
eval (Conv       e1 e2)  = undefined -- TODO
eval (Shift tau e)       = \t -> (shift tau) (eval e) t -- shifts e by tau, i.e. begins after tau seconds
eval (Negate e)          = negate (eval e)
eval (Exp e)             = exp    (eval e)
eval (Sin e)             = sin    (eval e)
eval (Cos e)             = cos    (eval e)
eval (Tan e)             = tan    (eval e)
eval (Asin e)            = asin   (eval e)
eval (Acos e)            = acos   (eval e)
eval (Atan e)            = atan   (eval e)
eval (Integral   e e0)   = undefined -- TODO
eval (Derivative e e0)   = undefined -- TODO
\end{code}
\end{codefig}



\begin{codefig}
\todo[inline]{Note: I'm not certain the Shift:s are correct or whether there should be a
  negate there somewhere, thus I've done TODO:s on all I'm not sure get the
  correct sign. Also, this is still in development, needs cleanup.} ~\\
\caption{Definition of \cmd{laplace}, the Laplace transform of an \cmd{Expression}.}\label{code:laplace-def}
\begin{code}
-- rules
laplace :: Expression Complex -> Expression Complex -- TODO: maybe instead do Time->Freq
laplace (e1 :+: e2) = laplace e1 :+: laplace e2 -- Superposition
laplace (e1 :-: e2) = laplace e1 :+: Negate (laplace e2) -- special case of superposition
laplace (Const a :*: e) = Const a :*: laplace e -- scaling
laplace (Negate e) = Negate (laplace e) -- can be seen as scaling with -1 as well
laplace (Derivative e e0) = Id :*: laplace e :-: (Const e0) -- L {f'} = \s -> s * L {f} - f(0)
laplace (Integral e e0) = laplace e :/: Id -- L {int_0^t f(x) dx} = s -> L {f} /s 
laplace (Shift tau e) = Exp (Id :*: Const tau) :*: laplace e -- TODO: double check signs
  -- TODO: double check signs on next two
laplace (Exp (Const tau :*: Id) :*: e) = Shift (negate tau) (laplace e) -- semantically same as next one
laplace (Exp (Negate (Const tau :*: Id)) :*: e) = Shift tau (laplace e)
laplace (Conv e1 e2) = laplace e1 :*: laplace e2 -- convolution turns into multiplication
-- transforms
laplace (Impulse) = Const 1 -- L {delta(t)} = \s -> 1
laplace (Const a) = Const a :/: Id -- L {a} = \s -> a/s; a = 1 => L {1} = 1/s
laplace (Id)      = Const 1 :/: (Id :*: Id) -- L {t} = \s -> 1/(s^2)
laplace (Exp (Negate (Const a :*: Id))) = Shift (negate a) (Const 1 :/: Id)-- TODO: Check signs
laplace (Exp (Const a :*: Id)) = Shift a (Const 1 :/: Id)-- TODO: Check signs
laplace (Const 1 :-: Exp (Negate (Const a :*: Id))) = Const a :/: (Id :*: Shift a Id) -- TODO: Doublecheck signs. Also, is this necessary?

laplace (Exp Id) = laplace (Exp (Const 1 :*: Id))
-- ... some more special cases, not sure we care?
-- TODO: Doublecheck signs on the next two
laplace (Exp (Negate (Const a :*: Id)) :*: Sin (Const omega :*: Id)) = Const omega :/: (Shift a (Id :*: Id) :+: (Const (omega*omega)))
laplace (Exp (Negate (Const a :*: Id)) :*: Cos (Const omega :*: Id)) = (Shift a Id) :/: (Shift a (Id :*: Id) :+: (Const (omega*omega)))
\end{code}
\end{codefig}

%The following code contains functions that print out an Expression in more human readable code. It's lightly adapted from \cite{TSSarbete}.
% \begin{code}
% -- -- | Show instance for Showable Expressions
% -- instance Show a => Show (Expression a) where
% --          show = showE
% 
% showE :: (Show a) => Expression a -> String
% showE exp = "f(var) = " ++ showExp exp
% 
% showExp :: (Show a) => Expression a -> String
% showExp (Const a)   = show a
% showExp Id          = "id(var)"
% showExp Pi          = "pi"
% showExp Impulse     = "delta(var)"
% showExp (Exp e)     = "e^(" ++ showExp e ++ ")"
% showExp (Sin e)     = "sin (" ++ showExp e ++ ")"
% showExp (Negate exp) = "-" ++ showExp exp 
% showExp (e0 :*: e1) = showCompExps e0 ++ "" ++ showCompExps e1
% showExp (e0 :+: e1) = showCompExps e0 ++ " + " ++ showCompExps e1
% showExp (e0 :-: e1) = showCompExps e0 ++ " - " ++ showCompExps e1
% showExp (e0 :/: e1) = showCompExps e0 ++ " / " ++ showCompExps e1
% showExp (Conv e0 e1) = showCompExps e0 ++ " * " ++ showCompExps e1
% showExp (Shift offset exp) = showExpWithOffset offset exp
% 
% -- | Composed Expressions needs to be enclosed in a pair of parens
% showCompExps :: (Show a) => Expression a -> String
% showCompExps (e0 :+: e1) = "(" ++ showCompExps e0 ++ " + " ++ showCompExps e1 ++ ")"
% showCompExps (e0 :-: e1) = "(" ++ showCompExps e0 ++ " - " ++ showCompExps e1 ++ ")"
% showCompExps (e0 :*: e1) =  "(" ++ showCompExps e0 ++ ")(" ++ showCompExps e1 ++ ")"
% showCompExps (e0 :/: e1) = "(" ++ showCompExps e0 ++ " / " ++ showCompExps e1 ++ ")"
% showCompExps exp         = showExp exp
% 
% showExpWithOffset :: (Show a) => a -> Expression a -> String
% showExpWithOffset offset Id      = "(var - " ++ show offset ++ ")"
% showExpWithOffset offset Impulse = "delta(var - " ++ show offset ++ ")"
% showExpWithOffset _ Pi           = "pi"
% showExpWithOffset _ (Const a)    = show a
% showExpWithOffset offset exp     = showExp (traverseExpE (Shift offset) exp)
% 
% traverseExpE e (e0 :*: e1) = (e e0) :*: (e e1)
% traverseExpE e (e0 :+: e1) = (e e0) :+: (e e1)
% traverseExpE e (e0 :-: e1) = (e e0) :-: (e e1)
% traverseExpE e (e0 :/: e1) = (e e0) :/: (e e1)
% traverseExpE e exp         = e exp
% 
% -- | Minimerar en Expression datatyp tills tva minimeringar returernar
% -- samma värde
% minE :: (Num a, Eq a) => Expression a -> Expression a
% minE e
%      | e == e' = e
%      | otherwise = minE e'
%      where e' = minE' e
% 
% -- | Minimerar en Expression-datatyp genom att ta bort onödiga termer.
% minE' :: (Num a, Eq a) => Expression a -> Expression a
% minE' (e :*: Const 1) = minE' e
% minE' (Const 1 :*: e) = minE' e
% minE' (Const a :*: Const b) = Const (a * b)
% minE' (e0 :*: e1) = minE' e0 :*: minE' e1
% minE' (e :*: Const 0) = Const 0
% minE' (Const 0 :*: e) = Const 0
% minE' (e :/: Const 1) = minE' e
% minE' (Const 1 :/: e) = minE' e
% minE' (e0 :/: e1) = minE' e0 :/: minE' e1
% 
% minE' (Const a :+: Const b) = Const (a+b)
% minE' (e :+: Const 0) = minE' e
% minE' (Const 0 :+: e) = minE' e
% minE' (e0 :+: e1) = minE' e0 :+: minE' e1
% minE' (e :-: Const 0) = minE' e
% minE' (Const 0 :-: e) = minE' e
% minE' (e0 :-: e1) = minE' e0 :-: minE' e1
% 
% minE' (Shift 0 e) = minE' e
% minE' (Exp (Const 0)) = Const 1
% minE' (Negate (Negate e)) = minE' e
% minE' (Negate e) = Negate (minE' e)
% minE' (Exp e) = Exp (minE' e)
% minE' (Sin e) = Sin (minE' e)
% minE' (Cos e) = Cos (minE' e)
% minE' expr = expr
% \end{code}
